\RequirePackage{fix-cm}

%\documentclass[10pt,times,authoryear]{article}
\documentclass[UTF8]{ctexart}
\usepackage{hyperref}
\usepackage{amsmath}
\usepackage{amssymb}
\usepackage{graphicx}
\usepackage{multirow}
\usepackage{booktabs}
\usepackage{algorithm}
\usepackage{algorithmicx}
\usepackage{subfig}
\usepackage[font=footnotesize]{caption}
\usepackage{url}
\usepackage[figuresright]{rotating}
\usepackage{booktabs}
\usepackage{amsmath}
%\usepackage{CJKutf8}

\usepackage[a4paper,left=1.5cm,right=1.5cm,top=1.5cm,bottom=1.9cm,bindingoffset=5mm]{geometry}

\usepackage{array}
\usepackage{tabularx}
\newcommand{\tabincell}[2]{\begin{tabular}{@{}#1@{}}#2\end{tabular}}
\newcommand{\loes}[2]{\medskip\noindent{\color{blue}{{\bf Lösung zu Aufgabe #1}\\ {#2}}}\vspace{0.5cm}}
\usepackage{graphicx}
\usepackage{pythonhighlight}

\usepackage{lineno}
\modulolinenumbers[5]


\newtheorem{definition}{Definition}
\newtheorem{task}{Task}

\begin{document}

\pagestyle{empty}
%\begin{flushright}
%Semester: 2018 fall\\
%Course No.: ?????????????????\\ %B1A351031
%Course:计算机科学与程序设计\\
%\end{flushright}  

\pagestyle{plain}
\setcounter{page}{1}
%\pagenumbering{roman}




\vspace{3cm}
\begin{center}
\Large{考试试卷}
\textbf{\Large{\\
Examination Paper A
}}

\vspace{1cm}
\Large{Fall Semester of 2020-2021 Academic Year }\\
\vspace{1cm}
\Large{Introduction to Computer Science and Programming}\\
\vspace{0.3cm}
\Large{Midterm Examination}
\vspace{1cm}
\end{center}


\vspace{3cm}
{\renewcommand{\arraystretch}{2.9}%
\begin{center}
    \begin{tabular}{rl}
    班  级 (Class):          &  \_\_\_\_\_\_\_\_\_\_\_\_\_\_\_\_\_\_\_\_\_\_\_\_\_\_\_\_\_\_\_\_\_\_\_\_\_\_\_\_\_\_\_                               \\
    学  号 (Student Number): &   \_\_\_\_\_\_\_\_\_\_\_\_\_\_\_\_\_\_\_\_\_\_\_\_\_\_\_\_\_\_\_\_\_\_\_\_\_\_\_\_\_\_\_                               \\
姓  名 (Name):   & \_\_\_\_\_\_\_\_\_\_\_\_\_\_\_\_\_\_\_\_\_\_\_\_\_\_\_\_\_\_\_\_\_\_\_\_\_\_\_\_\_\_\_                               \\
        成  绩 (Score):           & \_\_\_\_\_\_\_\_\_\_\_\_\_\_\_\_\_\_\_\_\_\_\_\_\_\_\_\_\_\_\_\_\_\_\_\_\_\_\_\_\_\_\_                               \\
    \end{tabular}

\end{center}
}

\vspace{2cm}

{\renewcommand{\arraystretch}{2.9}%
\begin{center}


November 9th, 2020\\
\LARGE\bf{School of General Engineering}
\end{center}
}

\newpage

%\begin{center}
%\Large\bf{School of General Engineering Examination Paper}
%\end{center}

Examiners use only:
\begin{center}
{\renewcommand{\arraystretch}{2.5}%
\begin{tabular}{|r|c|c|c|c|c|c|c|c|c|}
\hline
      & Task 1 & Task 2 & Task 3 & Task 4  & Task 5 & Task 6 & Task 7 & Task 8 \\
\hline
    Score & ~      & ~      & ~      & ~       & ~  & ~ & ~ & ~          \\
\hline
\hline
      & Task 9 & Task 10 & Task 11 & Task 12 & Task 13 & Task 14 & Task 15 &  Task 16   \\
\hline
    Score & ~      & ~      & ~      & ~       & ~  & ~ & ~ & ~  \\
\hline
    \end{tabular}
    }
\end{center}

\vspace{2cm}
Total Score (max. 100):  \_\_\_\_\_\_\_\_\_\_\_\_\_\_\_\_\_\_\_\_\_\_\_\_\_\_\_\_\_\_

\vspace{1cm}
Remarks:
\begin{itemize}
 \setlength{\itemsep}{0pt}
\item \textbf{This examination is \textit{closed-book} and \textit{paper-pencil/pen}. That means, you are \textit{not} allowed to use any additional material!}
\item \textbf{In total, you have 90 minutes for solving all 16 tasks.}
\end{itemize}


%\newpage
%
%Task overview (will not be shown in the final exam sheet!)
%
%\begin{enumerate}
%\item Category: Python program tracing (five minutes each):
%\begin{description}
%\item \textbf{Task~1}: Tracing: reverse list 
%\item \textbf{Task~2}: Tracing: convert decimal number to binary number 
%\item \textbf{Task~3}: Tracing: 01 sequence 
%\item \textbf{Task~4}: Tracing: double Pascal triangle
%\item \textbf{Task~5}: Tracing: max of three 
%\end{description}
%\item Category: Writing own Python program for given problem (five minutes each):
%\begin{description}
%\item \textbf{Task~6}: Problem three term Fibonacci
%\item \textbf{Task~7}: Problem GCD
%\item \textbf{Task~8}: Problem Max and Min 
%\item \textbf{Task~9}: Problem Dichotomy
%\end{description}
%\item Category: Debugging Python programs (five minutes each):
%\begin{description}
%\item \textbf{Task~10}: Debugging addition
%\item \textbf{Task~11}: Debugging prime
%\item \textbf{Task~12}: Debugging rectangle class
%\item \textbf{Task~13}: Debugging word dictionary
%\item \textbf{Task~14}: Debugging insertion sort 
%\end{description}
%
%\item Category: Complex task (15 minutes for task):
%\begin{description}
%\item \textbf{Task~15}: Choose two of the five Numbers and list all the choices
%\end{description}
%\end{enumerate}

\newpage

\begin{task}
{\textbf{Tracing Python programs}} (5 points / 5 minutes): Please write down the output of the following Python code.
\end{task}
\begin{verbatim}
    def f(L):
        if len(L)==1:
            return L
        else:
            return f(L[1:])+[L[0]]
    print(f([1,5,4,3]))
\end{verbatim}

\noindent
\textbf{Output of the code:}
\vspace{1cm}

\begin{task}
{\textbf{Tracing Python programs}} (5 points / 5 minutes): Please write down the output of the following Python code.
\end{task}
\begin{verbatim}
    def f(n):
        res=''
        while(n>0):
            res=str(n%2)+res
            n=n//2
            print(res)
        return int(res)
    f(10)
\end{verbatim}

\noindent
\textbf{Output of the code:}
\vspace{2cm}

\begin{task}
{\textbf{Tracing Python programs}} (5 points / 5 minutes): Please write down the output of the following Python code.
\end{task}
\begin{verbatim}
    def f(n):
        if n==1:
            return [[1],[0]]
        elif n>1:
            res=[]
            for L in f(n-1):
                res.append(L+[0])
                res.append(L+[1])
            return res
    print(f(3))
\end{verbatim}

\noindent
\textbf{Output of the code:}
\vspace{1cm}

\newpage
\begin{task}
{\textbf{Tracing Python programs}} (5 points / 5 minutes): Please write down the output of the following Python code.
\end{task}

\begin{verbatim}
    def f(n,L):
        if len(L)<n:
            X=[1]*(len(L)+1)
            for i in range(0,len(L)-1):
                X[i+1]=2*(L[i]+L[i+1])
            f(n,X)
        print (L)
    f(4,[1])
\end{verbatim}

\noindent
\textbf{Output of the code:}

\vspace{2cm}

\begin{task}
{\textbf{Tracing Python programs}} (5 points / 5 minutes): Please write down the output of the following Python code. Please note that True is equivalent to 1 and False is equivalent to 0 in numerical Python expressions.
\end{task}
\begin{verbatim}
def f(a,b,c):
    return (((b>a)*a+(b<a)*b)>c)*c+(((b>a)*a+(b<a)*b)<c)*((b>a)*a+(b<a)*b)
print(f(5,6,10))
\end{verbatim}

\noindent
\textbf{Output of the code:}
\vspace{1cm}

\begin{task}
{\textbf{Writing a Python program}} (5 points / 5 minutes): Please write a function to find the maximum value of a given list. 
\end{task}

\noindent
\textbf{Your Python program:}
\vspace{4cm}

\begin{task}
{\textbf{Writing a Python program}} (5 points / 5 minutes): Given $a_0=0,a_1=1,a_2=2$, $a_n=a_{n-1}+a_{n-2}+a_{n-3}~(n>2,n\in N)$. Please write a recursive Python function to calculate $a_n$.
\end{task}

\noindent
\textbf{Your Python program:}
\vspace{4cm}

\begin{task}
{\textbf{Writing a Python program}} (5 points / 5 minutes): Please write a recursive Python function to calculate the greatest common divisor of two numbers.
\end{task}

\noindent
\textbf{Your Python program:}
\vspace{4cm}





\begin{task}
{\textbf{Writing a Python program}} (5 points / 5 minutes):
\end{task}
\begin{enumerate}
    \item Write a function $f(x)=2x^2+x-2$. For an input \emph{x}, \emph{f(x)} should be output.
    \item Please use dichotomy to obtain two approximate roots with the accuracy 0.001 in the range \emph{(0,1)}.
\end{enumerate} 

\noindent
\textbf{Your Python program:}
\vspace{7cm}


\begin{task}
{\textbf{Writing a Python program}} (5 points / 5 minutes): Please write a function to check whether a given number $n$ can be written as the product of two smaller, even numbers $a$ and $b$.
\end{task}

\noindent
\textbf{Your Python program:}
\vspace{6cm}


\newpage
%%%%%%%%%%%%%%%%%%%%%%%%%%%%%%%

\begin{task}
{\textbf{Debugging a Python program}} (5 points / 5 minutes): Please debug the following program to achieve the desired output. There are two errors!
\end{task}

\begin{verbatim}
    def addition(a,b):
        res=a+b
        return res
    def subtraction(a,b):
        result=a-b
        print(result)
    RES=addition(1,1)
    print("The result of addition is",res)
    print("The result of subtraction is",subtraction(2,1))
\end{verbatim}
\textbf{The desired output of this program should be:}
\begin{verbatim}
    The result of addition is 2
    The result of subtraction is 1
\end{verbatim}
%2

\vspace{1cm}

%%%%%%%%%%%%%%%%%%%%%%%%%%%%%%%

\begin{task}
{\textbf{Debugging a Python program}} (5 points / 5 minutes): Please debug the following program to get all the prime numbers from $1$ to $20$. There are three errors!
\end{task}
\begin{verbatim}
    def checkPrime(data,n=2):
        if data==1:
            return False
        elif data==n:
            return True
        else:
            if data%n!=0:
                checkPrime(data,n-1)
            return False
    def getPrimes(i,j):
        L=[]
        for k in range(i,j+1):
            if checkPrime(k)=True:
                L.append(k)
            return L
    print(getPrimes(1,20))
\end{verbatim}
\textbf{The desired output of this program should be:}
\begin{verbatim}
    [2,3,5,7,11,13,17,19]
\end{verbatim}
%3

%%%%%%%%%%%%%%%%%%%%%%%%%%%%%%%
\newpage
\begin{task}
{\textbf{Debugging a Python program}} (5 points / 5 minutes): Please debug the following program to compare the size of two rectangles. There are three errors!
\end{task}
\begin{verbatim}
    class Rectangle:
        def __init__(length,width,name):
            self.length=length
            self.width=width
            self.name=name
        def get_area(self):
            return self.length*self.width
        def compare_size(self,other):
            if self.get_area<other.get_area:
                print(other.name,"is larger")
            else:
                print(self.name,"is larger")
    R1=Rectangle(1,2,"R1")
    R2=Rectangle(1,3,"R2")
    compare_size(R1,R2)
\end{verbatim}
\textbf{The desired output of this program should be:}
\begin{verbatim}
    R2 is larger
\end{verbatim}
%3

\vspace{1cm}

%%%%%%%%%%%%%%%%%%%%%%%%%%%%%%%

\begin{task}
{\textbf{Debugging a Python program}} (5 points / 5 minutes): Please debug the following program to count the occurrence of words. There are two errors!
\end{task}
\begin{verbatim}
    word_list=["a","b","b","c","c","c"]
    count_dict={}
    for word in word_list:
        if word in count_dict:
            count_dict[word]=1
        else:
            count_dict[word]+=1
    print(count_dict)
\end{verbatim}
\textbf{The desired output of this program should be:}
\begin{verbatim}
    {'a': 1, 'b': 2, 'c': 3}
\end{verbatim}
%2

%%%%%%%%%%%%%%%%%%%%%%%%%%%%%%%
\newpage
\begin{task}
{\textbf{Debugging a Python program}} (5 points / 5 minutes): Please debug the following program to sort the list with insertion sort algorithm. There are two errors!
\end{task}
\begin{verbatim}
    def InsertionSort(L):
        for i in range (len(L)):
            j=i-1
            while j>=0 and L[j]>L[i]:
                L[j+1]=L[j]
                j-=1
                L[j+1]=L[i]
        return L
    print(InsertionSort([5,4,3,2,1]))  
\end{verbatim}
\textbf{The desired output of this program should be:}
\begin{verbatim}
    [1,2,3,4,5]
\end{verbatim}
%2

\newpage

\begin{task}
{\textbf{Complex task}} (15 points / 15 minutes): Please write down the output of the following Python code.
\end{task}
\begin{verbatim}
    x=5
    y=2
    l=[]
    for i in range(y):
        l.append(1)
    p=0
    while(p>-1):
        if(l[p]>x):
            p-=1
        else:
            if(p==y-1):
                print(l)
            else:
               l[p+1]=l[p]
               p=p+1
        l[p]+=1
\end{verbatim}

\newpage
Additional space for your answers:
\mbox{~}
\clearpage


\end{document}

